
\section{Implementation}
\label{implement}

\dejavu{} is implemented in \scalalang{}.   
\dejavu{} takes as input a specification file containing one  or 
more  properties, and synthesizes the monitor as a self-contained \scalalang{} 
program.
This program takes as input the 
trace file and analyzes it.
The tool uses the JavaBDD library for BDD manipulations \cite{javabdd}.

\begin{figure}
\begin{center}
\begin{lstlisting}[language=dsl,frame=single,linewidth=0.95\textwidth,backgroundcolor=\color{white},linewidth=\columnwidth,breaklines=true,basicstyle=\small]
prop telemetry1: Forall x . closed(x) -> !telem(x) where 
  closed(x) := toggle(x) <-> @!closed(x) 

prop telemetry2: Forall x . closed(x) -> !telem(x) where
  closed(x) := (!@true & !toggle(x)) | (@closed(x) & !toggle(x)) | 
               (@open(x) & toggle(x)),
  open(x) := (@open(x) & !toggle(x)) | (@closed(x) & toggle(x))

prop spawning : Forall x . Forall y . Forall d . report(y,x,d) -> spawned(x,y) where
  spawned(x,y) := @spawned(x,y) | spawn(x,y) | 
                  Exists z . (@spawned(x,z) & spawn(z,y))

prop commands : Forall c . dispatch(c) -> ! already_dispatched(c) where
  already_dispatched(c) := @ [dispatch(c) , complete(c)),
  dispatch(c) := Exists t . CMD_DISPATCH(c,t),
  complete(c) := Exists t . CMD_COMPLETE(c,t)
\end{lstlisting}
\caption{Properties stated in \dejavu's logic}
\label{fig:properties}
\end{center}
\end{figure}

\vspace{1ex}\noindent {\bf Example properties.}
Figure \ref{fig:properties} shows four properties in the 
input ASCII format of the tool, the first three of which are related 
to the examples in Section~\ref{sec:syntax-semantics}, which
are not expressible in (P)\FLTL{}. That is,
these properties are not expressible in the original first-order 
logic of \dejavu{} presented in \cite{HPU}. The last property illustrates the use of rules to perform conceptual abstraction.
%
The ASCII version of the logic uses 
\idsl{@} for $\ominus$,
\idsl{|} for $\vee$,
\idsl{&} for $\wedge$, and
\idsl{!} for $\neg$.
%
The first property \idsl{telemetry1} is a variant of
formula \ref{eq:wolper-first-order}, illustrating the use of a 
rule to express a first-order version of Wolper's example \cite{Wolper}, that all the states with even indexes of a sequence satisfy a property.
In this case we consider a radio on board a spacecraft, which communicates over different channels (quantified over in the formula) that can be turned on and off with a \idsl{toggle(x)}; they are initially off.
Telemetry can only be sent to ground over a channel \idsl{x} with the \idsl{telem(x)} event when radio channel \idsl{x} is toggled on.

The second property, \idsl{telemetry2}, expresses the same 
property as \idsl{telemetry1}, but in this case using two 
rules, reflecting how we would model this using a state machine with two states for each channel \idsl{x}: 
\idsl{closed(x)} and \idsl{open(x)}. The rule \idsl{closed(x)} is defined as a disjunction between three alternatives. The first states that this predicate is true if we are in the initial state 
(the only state where \idsl{@true} is false), and there is no \idsl{toggle(x)} event. The next alternative states that
\idsl{closed(x)} was true in the previous state and there is
no \idsl{toggle(x)} event.
The third alternative states that in the previous state we were
in the \idsl{open(x)} state and we observe a \idsl{toggle(x)} 
event. Similarly for the \idsl{open(x)} rule.
 
The third property, \idsl{spawning}, expresses a property
about threads being spawned in an operating system. We want to 
ensure that when a thread \idsl{y} reports some data \idsl{d} 
back to another thread \idsl{x}, then thread \idsl{y} has been spawned by thread \idsl{x} either directly, or transitively 
via a sequence of spawn events. The events are
\idsl{spawn(x,y)} (thread \idsl{x} spawns thread \idsl{y})
and \idsl{report(y,x,d)} (thread \idsl{y} reports data \idsl{d}
back to thread \idsl{x}).
For this we need to compute a transitive closure of spawning relationships, here expressed with the rule \idsl{spawning(x,y)}.

The fourth property, \idsl{commands}, 
concerns a realistic log from from the Mars rover Curiosity 
\cite{msl}. The log consists of events 
(here renamed) \idsl{CMD_DISPATCH(c,t)} and \idsl{CMD_COMPLETE(c,t)}, 
representing the dispatch and subsequent completion  of a command 
\idsl{c} at time \idsl{t}. The property to be verified is that a command, once 
dispatched, is not dispatched again before completed. Rules 
are used to break down the formula to conceptually simpler pieces. 
%The example
%illustrates how rules elegantly replace the role of macros, a non-trivial extension 
%introduced in an earlier version of \dejavu.

% Out of 25,463 dispatches and 23,740 completions, there were 2,466
% violations of this property.
%
% A formulation of this property without using rules is:
%
% prop commands :
%   Forall c .
%     (Exists t1 . CMD_DISPATCH(c,t1)) -> 
%       ! @ [
%             Exists t2 . CMD_DISPATCH(c,t2), 
%             Exists t3 . CMD_COMPLETE(c,t3)
%           )   

\iffalse
\subsubsection{Monitor Synthesis}

Consider the first \idsl{telemetry1} property. The property-specific 
part of the synthesized monitor, is shown 
in Figure \ref{fig:monitor} (left)\footnote{An additional 600+ lines of property independent 
boilerplate code is generated.}. This function is called for each new event. The function operates on the two arrays holding BDDs, indexed by subformula indexes: \iscala{pre} for the previous state and \iscala{now} for the current state. For each observed event, the function \iscala{evaluate()} computes the \iscala{now} array, evaluating any sub-formula before any formula containing it,
and returns true (property is satisfied in this position of the trace)  iff \idsl{now(0)} is not $\bfalse$. 

\begin{figure}[htb]
\centering
\begin{minipage}[b]{.51\textwidth}
\begin{lstlisting}[language=scala,basicstyle=\sffamily]
override def evaluate(): Boolean = {
  // -- 1. rule nodes not below @:
  now(7) = pre(8)
  now(6) = build("toggle")(V("x"))
  now(5) = now(6).biimp(now(7))
  // -- 2. rule calls:
  now(9) = now(5)
  now(2) = now(5)
  // -- 3. rest of rules below @:
  now(8) = now(9).not()
  // -- 4. rest of main formula:
  now(4) = build("telem")(V("x"))
  now(3) = now(4).not()
  now(1) = now(2).not().or(now(3))
  now(0) = now(1).forAll(var_x.quantvar)

  val error = now(0).isZero
  if (error) monitor.recordResult()
  tmp = now; now = pre; pre = tmp
  !error
}
\end{lstlisting}
\end{minipage}
\qquad
\begin{minipage}[b]{.43\textwidth}
\includegraphics[width=7cm]{figures/ast.pdf}
\end{minipage}
\caption{Monitor (left) and subformula enumeration (right) for property telemetry1}
\label{fig:monitor}
\end{figure}

The indexing scheme is guided by an enumeration of the annotated abstract syntax tree of the formula, shown in Figure \ref{fig:monitor} (right). The figure (generated by \dejavu) points out (blue node 5) the top node for the rule \idsl{closed(x)}, and calls to the rule (purple nodes 2 and 9).
The subformulas are evaluated in four steps, as outlined at the end of Section \ref{LTLruntime}. First rule nodes not below the previous-operator \idsl{@} (7, 6, 5). Then calls of rules (9, 2)
which can now be calculated since the rule bodies have been calculated. Then the rest of the rules below \idsl{@}  (8). Finally the rest of the main formula (4, 3, 1, 0). At composite subformula nodes, BDD operators are applied. For example for subformula  1, the new  value is \iscala{now(2).not().or(now(3))}, 
which is the interpretation of the formula 
\idsl{closed(x) -> !telem(x)} corresponding to the Boolean law:
$p \rightarrow q = \neg p \vee q$.

\subsubsection{Example Execution}

As an example of how a trace is processed consider the following simple (correct) trace containing two events, turning on low and high frequency channels:
$\mathit{toggle}(``low").\mathit{toggle}(``high")$. The value $``low"$ is assigned the binary enumeration $110$ in node 6, shown 
as a BDD in Figure \ref{fig:bdd1}. A dashed line leaving a node means the node has value 0, and a fully drawn line means it has value 1. The BDD represents all assignments to the bit positions 0, 1 and 2 that lead to 1 (shaded square leaf node), in this case 110, read from left to right. Likewise, Figure \ref{fig:bdd2} shows the BBD assigned to node 6 for the second value $``high"$, corresponding to the enumeration 101. Finally Figure \ref{fig:bdd3} shows the BDD representing the rule \idsl{closed(x)} in node 5 after the two events. Since channels $``low"$ and $``high"$, corresponding to enumerations 110 and 101, are now open, this BDD represents all other enumerations. That is, all enumerations except 110 and 101 lead to 1 in this BDD.

\begin{figure}
    \centering   
    \begin{subfigure}[b]{0.25\textwidth}
        \includegraphics[width=\textwidth]{figures/bdd1.pdf}
        \caption{toggle(``low'')}
        \label{fig:bdd1}
    \end{subfigure}
    ~ 
    \begin{subfigure}[b]{0.25\textwidth}
        \includegraphics[width=\textwidth]{figures/bdd2.pdf}
        \caption{toggle(``high'')}
        \label{fig:bdd2}
    \end{subfigure}
    ~ 
    \begin{subfigure}[b]{0.25\textwidth}
        \includegraphics[width=\textwidth]{figures/bdd3.pdf}
        \caption{closed(x)}
        \label{fig:bdd3}
    \end{subfigure}
    \caption{Selected BDDs from trace evaluation}
    \label{fig:bdds}
\end{figure}

\fi

\vspace{1.5ex}\noindent {\bf Evaluation.}
In \cite{HPU,HPU-FMSD} we evaluated \dejavu{}
without the rule extension against the \monpoly{} tool \cite{agrebasin}, which supports a logic close to \dejavu's. In \cite{HP} we evaluated \dejavu's garbage collection capability. In this section we evaluate the rule extension for the properties in Figure 
\ref{fig:properties} on a collection of 
traces. Table \ref{tab:evaluation} shows the  
analysis time (excluding time to compile the 
generated monitor) and maximal memory usage in 
MB for different 
% ---
traces
%\footnote{
%For the first two properties the same traces were used, generated as follows for varying constants $P$, $Q$ and $R$.
%Repeat $P$ times the following: toggle $Q$ 
%channels; then for each channel send telemetry $R$ %times; then toggle the channels again.
%For the third property, traces were 
%generated as follows for varying constants $S$ 
%and $T$: spawn $S$ threads from the main thread 0, and %let each report back to thread 0; then repeat $T$ times %the following: each newly spawned thread in previous %step
%spawns a new thread, which reports back to the main %thread 0.
%}
% ---
(format is `trace length : time / memory'). 
%
%The telemetry traces will repeatedly open 
%hundreds of channels and send telemetry over 
%them. 
%
%The spawning traces consist of 50\% spawning 
%events and 50\% reporting events. 
%
The processing time is generally
very reasonable for very large traces. However,
the spawning property  requires considerably larger 
processing time and memory compared to the other 
properties since more data (the transitive 
closure) has to be computed and stored.
%
The evaluation was performed on a Mac laptop, with the Mac OS X 10.10.5 operating 
system, on a 2.8 GHz Intel Core i7 with 16 GB of memory.

\ignore{
  // repeat: repeat the following
  // repeat_toggle: open this many channels
  // repeat_telem: send this many messages on each channel
}

\ignore{
  // threads: how many threads initially spawned by main program
  // repeat: how many times new threads from new threads are spawned
  // number of spawned threads correspond to number of events
}
\begin{table}[h!]
\centering
\begin{tabular}{|l|c|c|c|}
\hline
Property   & Trace 1 & Trace 2 & Trace 3 \\
\hline
telemetry1 & 1,200,001\;:\;2.6s / 194 MB & 5,200,001\;:\;5.9s / 210 MB & 10,200,001\;:\;10.7s / 239 MB \\
\hline
telemetry2 & 1,200,001\;:\;3.8s / 225 MB & 5,200,001\;:\;8.7s / 218 MB & 10,200,001\;:\;16.6s / 214 MB \\
\hline
spawning   & 9,899\;:\;29.5s / 737 MB & 19,999\;:\;117.3s / 1,153 MB & 39,799\;:\;512.5s / 3,513 MB \\
\hline
commands   & 49,999\;:\;1.5s / 169 MB & N/A & N/A \\
\hline
\end{tabular}
\caption{  Evaluation - trace lengths, analysis time in seconds, and maximal memory use}
\label{tab:evaluation}
\end{table}

%\begin{table}[h!]
%\centering
%\begin{tabular}{|l|c|c|c|}
%\hline
%Property   & Trace 1 & Trace 2 & Trace 3 \\
%\hline
%telemetry1 & 1,200,001\,:\,3.3s & 5,200,001\,:\,7.1s & 10,200,001\,:\,12.8s \\
%\hline
%telemetry2 & 1,200,001\,:\,4.4s & 5,200,001\,:\,9.9s & 10,200,001\,:\,17.9s \\
%\hline
%spawning   & 9,899\,:\,42.7s & 19,999\,:\,127.6s & 39,799\,:\,572.4s \\
%\hline
%\end{tabular}
%\caption{Evaluation - trace lengths and analysis time in seconds}
%\label{tab:evaluation}
%\end{table}

