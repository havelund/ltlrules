
\begin{abstract}

Runtime Verification (RV) consists of analyzing execution 
traces using formal techniques. 
%RV is scalable and 
%therefore has a big potential for infusion in software 
%engineering practices. 
It includes monitoring the 
execution of a system against properties formulated in 
Linear Temporal Logic (LTL). 
LTL offers a succinct notation for writing useful 
specification properties. However, it is limited in 
expressiveness in the propositional case, and several 
theoretic extensions have therefore been proposed. Furthermore, for many practical cases, there is a need to monitor properties 
that carry data, where one can use formalisms like 
first-order LTL. We show that 
first-order LTL has similar expressiveness
limitations as the propositional version. We suggest here 
two related extensions for increasing the expressive power: 
one for propositional LTL and one for 
first-order LTL. These extensions have a simple incremental operational 
semantics that is more suitable for RV than previously 
suggested extensions. We show that the 
propositional extension has the same expressiveness as the 
classical extensions, and demonstrate its
adoption for runtime verification of first-order safety 
properties. Finally, we expand the BDD-based runtime verification 
tool \dejavu{} to support our extension and perform some 
experiments.

\end{abstract}