
\section{Conclusions}

Propositional linear temporal logic (LTL) and automata are two common specification formalisms for software and hardware systems. While temporal logic has a more declarative
flavor, automata are more operational, describing how the specified system
progresses. There has been several proposed extensions to LTL that extend
its expressive power to that of related automata formalisms. We proposed here
a simple extension for propositional LTL that is based on adding propositions that summarize the prefix of an execution. Conceptually, this extension puts the specification in between propositional LTL and automata, as the additional
variables can be seen as representing the state of an automaton
that is synchronized with the temporal property. It is shown to have
the same expressive power as automata models, and is in particular
appealing for runtime verification of past (i.e., safety) temporal properties,
which already are based on summarizing the value of subformulas over
observed prefixes.

We then demonstrated that first-order linear temporal logic (FLTL), which
can be used to assert properties about systems with data, also has expressiveness deficiencies, and extended it with relations that summarize prefixes.
We showed that for the first-order case, unlike the propositional case,
this extension is not identical to the addition of dynamic (i.e., state dependent) quantification.

We presented a monitoring
algorithm for propositional past time temporal logic, extending a classical algorithm, and similarly presented an algorithm for 
first-order past temporal logic.
Finally we described the implementation of this extension in the \dejavu{} tool and provided experimental results.
